\documentclass[
	fullscreen=true, 
	bookmarks=false,
	sans serif,
	9pt,
	pdf,
	hyperref={
		pdfpagelabels=false,
		unicode=true
	}
]{beamer}
\usepackage[T2A]{fontenc}
\usepackage[utf8]{inputenc}
\usepackage[english,russian]{babel}
\usepackage{pgfpages}
\pgfpagesuselayout{resize to}[a4paper,landscape,border shrink=0mm]

\usepackage{array}
\usepackage{epsfig}
\usepackage{epstopdf}
\usepackage{multirow}
\usepackage{listings}
\usepackage{tikz}
\graphicspath{ {images/} }

\lstset{
	tabsize=4,
	gobble=12,
	keepspaces=true,
	basicstyle=\normalsize\ttfamily,
	breaklines=true,
	columns=fullflexible,
}

\usetheme{Madrid}
% =====================================================================

% Info
\title{
	Разработка и исследование методов и алгоритмов обработки изображения по идентификации и обезличиванию объектов
}
\author{
	\vspace{0.5cm}\\
	\large
    Акимова Е. И.\\
	\vspace{0.1cm}
	УлГТУ\\
	\vspace{2cm}
}
\institute{}
% =====================================================================

% Document
\begin{document}
	
	% Title
	\begin{frame}
		\titlepage
	\end{frame}	
	% =====================================================================
		
	\section{}
	\subsection{}
		
	\begin{frame}\frametitle{Цель, объект и предмет исследования}
		Цель: улучшить качество распознавания и удаления текста с изображения при помощи нейронной сети.
		\newline
		
		Объект исследования: распознавание и удаление текста на изображении
		\newline
		
		Предмет исследования: распознавание и удаление текста на изображении при помощи нейронной сети.
	\end{frame}

	\begin{frame}\frametitle{Задачи и методы исследования}
		Задачи:
		\begin{itemize}
			\item изучение предметной области;
			\item генерация датасета;
			\item изучение методов распознования и удаления текста на изображении;
			\item выработка критериев успешности метода;
			\item эксперименты и вывод.
		\end{itemize}
		Методы:
		\begin{itemize}
			\item сравнительный анализ;
			\item изучение и анализ литературы;
			\item эксперимент;
			\item расчёты и измерения.
		\end{itemize}
	\end{frame}

	\begin{frame}\frametitle{Научная новизна}
		Научная новизна в исследовании заключается в разработке системы автоматического выделения и удаления текста на изображении, основанного на соединении двух нейронных сетей, одна из которых отвечает за выделение текста и создание маски, вторая - за удаление текста по маске.
	\end{frame}

	\begin{frame}\frametitle{Положения}
	На защиту выносятся:
		\begin{itemize}
			\item Метод выделения и удаления текста на изображении как синтез существующих методов;
			\item Модели системы, полученные в результате проектирования информационной системы выделения и удаления текста на изображении;
			\item Информационная система выделения и удаления текста на изображении.
		\end{itemize}
	\end{frame}

	\begin{frame}[fragile]\frametitle{Краткий обзор научных работ}	
		Научная статья  О.А. Строя и В.В. Буряченко посвящена удалению нежелательных объектов с аэроснимков. В ней рассматривают различные способы оработки изображения, одним из которых является нейронная сеть ResNet, которую легче оптимизировать и точность классификации которой повышается за счёт значительного увеличения глубины, при этом обучать её легче. Схема архитектуры данной сети:
		\begin{figure}[h]
            \centering
            \includegraphics[width=0.7\textwidth]{ResNet.jpg}
            \caption{Архитектура ResNet}
            \label{fig:1}
        \end{figure}
	\end{frame}
	
	\begin{frame}[fragile]\frametitle{Краткий обзор научных работ}	
        В следующей статье рассматривается метод реконструкции изображений на основе поиска подобных областей. Данный метод состоит из двух шагов. На первом шаге автоматизировано определяются поврежденные участки изображения или пользователем формируется маска дефектных областей. На втором шаге изображение с потерянными областями восстанавливается с помощью метода EBM (exemplarbasedmethod). Данный метод позволяет корректно восстанавливать потерянные области с помощью созданной маски, правильно реконструируя значения пикселей. Метод основывается на вычислении приоритета для каждого пикселя границы с последующим поиском похожего квадратного блока в области доступных пикселей и его копировании в область отсутствующих пикселей.
	\end{frame}
	
	\begin{frame}[fragile]\frametitle{Краткий обзор научных работ}	
        В статье "Разработка алгоритма сегментации рукописного текста с использованием маски" освещается проблема выделения рукописного текста на изображении для последующего распознавания. Приводится описание широко распространенных методов сегментации, позволяющих решить данную проблему: метод цветового анализа изображений, бинарная сегментация изображений методом фиксации уровня, а также метод выделения и анализа контуров. Предлогается алгоритм сегментации текста, основанный на использовании маски исходного изображения, позволяющий значительно упростить сегментацию текста на изображениях с немонотонным фоном. Также производится сравнительный анализ методов. Поиск различий маски и изображения проводился по формуле:
        \[ Diff(i, j) = Image(i, j) - Mask(i, j) \]
	\end{frame}
	
	\begin{frame}[fragile]\frametitle{Краткий обзор научных работ}	
        В статье "Использование технологии сверточных нейронных сетей в сегментации объектов изображения" речь идёт о компьютерном зрении, рассматривается применение нейронных сетей для выделения конкретных объектов на изображении, дана характеристика задачи сегментации и основных принципов компьютерного зрения.
	\end{frame}
	
	\begin{frame}[fragile]\frametitle{Краткий обзор научных работ}	
        Авторы статьи "Deep features based convolutional neural network model for text and non-text region segmentation from document images" представляют модель глубокой сверточной нейронной сети, которая использует функции глубокого обучения для сегментации текстовых и нетекстовых областей на изображениях документов. Основная задача - извлечение текстовых областей из изображений документов сложной компоновки без каких-либо предварительных знаний о сегментации. В реальном мире изображения документов или журналов содержат различную текстовую информацию наряду с нетекстовыми областями, такими как символы, логотипы, изображения и графика. Извлечение текстовых областей из нетекстовых является сложной задачей. Для решения этих проблем в данной работе предложена эффективная и надежная техника сегментации. Реализация предложенной модели разделена на три этапа: метод предварительной обработки изображений документов с использованием различных размеров патчей используется для обработки ситуаций, связанных с вариантами шрифтов и размеров текста на изображении; модель глубокой сверточной нейронной сети предлагается для предсказания текста, нетекста или неоднозначной области на изображении; метод постобработки изображения документа предлагается для обработки ситуации, когда изображение имеет сложные неоднозначные области, используя рекурсивное разбиение этих областей на соответствующие классы, затем система накапливает ответы этих предиктивных патчей с различным разрешением для обработки ситуации с вариациями текстовых шрифтов в изображении.
	\end{frame}
	
	\begin{frame}[fragile]\frametitle{Краткий обзор научных работ}	
        В статье "Distance transform based text-line extraction from unconstrained handwritten document images" речь идёт о новом методе извлечения текстовых строк для рукописных документов, не зависящем от языка, который справляется с такими сложностями, как соприкасающиеся и разнонаправленные текстовые строки, перекрывающиеся символы и неравномерные межстрочные интервалы. Данный метод осуществляет предварительную обработку страниц документа с помощью метода преобразования расстояний и использует новый алгоритм обнаружения контура для выделения отдельных текстовых строк. Предложенный метод был протестирован на шести стандартных наборах данных, которые находятся в открытом доступе, и результаты экспериментов показывают, что метод достигает многообещающей точности по сравнению с современными методами TLE.
	\end{frame}

	\begin{frame}\frametitle{}
		\Large
		\center
		\textbf{Спасибо!}\\
		\vspace{1.5cm}
		\normalsize
		Акимова Е. И.\\
		УлГТУ\\
		\vspace{0.5cm}
		\textit{shuey7399@mail.ru}
	\end{frame}
	
\end{document}
